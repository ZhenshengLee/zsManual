\documentclass[a4paper,twoside,cs4size,fancyhdr,notitlepage]{ctexart}
% 页面布局

\usepackage[top=2.54cm,bottom=2.54cm,left=3.17cm,right=3.17cm]{geometry} % 用于设置页面布局


% 字体和段落

    % 章节标题显示方式
\CTEXsetup[format={\Large\raggedright\bfseries}]{section}

    % 字体设置
% \usepackage{xeCJK}  % 用于使用本地字体


    %设置常用中文字号,方便调用


% 内容布局

    % 页眉页脚页码
\pagestyle{fancy}
\fancyhf{}
\lhead{黎振胜}
\rhead{\leftmark}
\cfoot{\thepage}

    % 表格和图片
\usepackage{amssymb}
\usepackage{amsmath}    % 在公式中用\text{文本}输入中文
\usepackage{diagbox}    % 斜线表
\usepackage{multirow}   % 表格中使用多行
\usepackage{booktabs}   % 三线表
\usepackage{rotating}   % 使用sidewaystable环境旋转表格
\usepackage{tabularx}   % 定宽表格
\usepackage{graphicx}   % 处理图片
\usepackage{footnote}   % 增强的脚注功能,可添加表格脚注
\usepackage{threeparttable} % 添加真正的表格脚注,示例见README
\usepackage{hyperref}   % 添加pdf书签


% 目录和文献

\usepackage[super, square, sort&compress]{natbib}   % 处理参考文献

% 代码格式
\usepackage{listings}   % 添加代码高亮
\lstset{
	basicstyle=\footnotesize,       % the size of the fonts that are used for the code
	numbers=left,                   % where to put the line-numbers
	numberstyle=\footnotesize,      % the size of the fonts that are used for the line-numbers
	numberstyle=\tiny,
	stepnumber=5,                  	% the step between two line-numbers. If it is 1 each line will be numbered
	numbersep=5pt,                  % how far the line-numbers are from the code
	% showspaces=false,            	% show spaces adding particular underscores
	% showstringspaces=false,      	% underline spaces within strings
	showtabs=false,                 % show tabs within strings adding particular underscores
	% frame=single,           		% adds a frame around the code
	tabsize=1,          			% sets default tabsize to 2 spaces
	breaklines=true,       			% sets automatic line breaking
	postbreak=\raisebox{0ex}[0ex][0ex]{\ensuremath{\color{red}\hookrightarrow\space}},
	showlines=true
}


% 命令重定义 
% \renewcommand{\refname}{\bfseries{参~考~文~献}} 	%将Reference改为参考文献(用于 article)
\renewcommand{\bibname}{参~考~文~献}	%将bibiography改为参考文献(用于 book)
\renewcommand{\baselinestretch}{1.38} 	%设置行间距
\renewcommand{\figurename}{\small\ttfamily 图}
\renewcommand{\tablename}{\small\ttfamily 表}

\title{P3AT机器人实验手册}
\author{黎振胜}
\date{\today}

\begin{document}
\maketitle{}

\section{必读}
    大家好,这是我离校之前对使用P3AT机器人的一些方法介绍和总结,不全面但都是一些很基本和重要的东西,使用机器人之前把这个浏览一遍可以少走弯路,希望对大家以后的实验有帮助。

    另外这算是我第二次使用\LaTeX 进行写作,权当作一次练习,希望大家以后也多多尝试。

    最后,希望后面的师弟能够把这份文档维护下去。既可以学习\LaTeX ,又可以学习Git,何乐而不为。
\subsection{基本使用}
\subsubsection{硬件}
灯提示及声音提示
\subsubsection{软件}
\subsection{充电}
\subsubsection{原装电池}
\subsubsection{高性能电池}
\subsection{里程计校准}
\subsection{资料清单}
\subsection{注意事项}
重新整理后,分文件夹进行介绍
\section{手柄控制}
\section{记录里程计信息}
\section{使用kinect}

\section{使用远程人机交互平台}
    正在逐步把所有软件安装到实验室白色服务器上,下面是我的记录\footnote{软件目录.../}
\subsection{软件安装}
    你们看到这个文档的时候环境应该已经配置好了,但我把它记下来以备不时之需。
\subsubsection{安装LabVIEW2014}
\begin{description}
    \item[安装软件] 大家都会装吧,只需注意一点,安装所有软件功能,咱们实验室的LabVIEW包含功能特别多,全部装上去
    \item[安装更新] 另外使用前最好彻底更新,方法是【帮助】-【检查更新】。网络条件不好,记住一个一个装,不然你会后悔
    \item[升级VI Package Manager] 升级到最新版本,当前是2016,目录为.../软件依赖项/labview2014/vipm-windows
    \item[安装附加包] 安装以下附加软件:
\end{description}

\appendix
\section{LabVIEW程序设计}
\subsection{LabVIEW调用C}
\subsection{LabVIEW调用Python}
\subsection{LabVIEW Actor Framework}
\subsection{LabVIEW与ROS}
\section{Linux基础}
\section{Git基础}
\section{ROS基础}



\end{document}

