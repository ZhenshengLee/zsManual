% 页面布局

\usepackage[top=2.54cm,bottom=2.54cm,left=3.17cm,right=3.17cm]{geometry} % 用于设置页面布局


% 字体和段落
	% 段落
\sloppy % 防止长文字超出右边边界,我主要是在输入路径中出现问题
\usepackage[hyphens]{url} % 解决脚注中太长的url超出右边界

    % 章节标题显示方式
\CTEXsetup[format={\Large\raggedright\bfseries}]{section}

    % 字体设置
\usepackage{xeCJK}  % 用于使用本地字体

	% 标签
\usepackage{hyperref}   % 添加pdf书签
\hypersetup{
	% colorlinks=true,	% 超链接渲染颜色
	pdfborder=0 0 1,
	pdfauthor=zhensheng, % pdf文件属性作者
	bookmarksnumbered=true	% 书签带编号
}
	% 脚注
\usepackage{footnote}   % 增强的脚注功能,可添加表格脚注
\usepackage{pifont}
\renewcommand\thefootnote{\ding{\numexpr171+\value{footnote}}}
    %设置常用中文字号,方便调用


% 内容布局

    % 页眉页脚页码
\pagestyle{fancy}
\fancyhf{}
\lhead{黎振胜}
\rhead{\leftmark}
\cfoot{\thepage}

    % 表格和图片
\usepackage{amssymb}
\usepackage{amsmath}    % 在公式中用\text{文本}输入中文
\usepackage{diagbox}    % 斜线表
\usepackage{multirow}   % 表格中使用多行
\usepackage{booktabs}   % 三线表
\usepackage{rotating}   % 使用sidewaystable环境旋转表格
\usepackage{tabularx}   % 定宽表格
\usepackage{graphicx}   % 处理图片
\usepackage{threeparttable} % 添加真正的表格脚注,示例见README  

	% 图表交叉引用
% \renewcommand{\figurename}{\small\ttfamily 图}
\renewcommand{\figureautorefname}{\small\ttfamily 图}
% \renewcommand{\tablename}{\small\ttfamily 表}
\renewcommand{\tableautorefname}{\small\ttfamily 表}


% 目录和文献

\usepackage[super, square, sort&compress]{natbib}   % 处理参考文献

% 代码格式
\usepackage{listings}   % 添加代码高亮
\lstset{
	basicstyle=\small\ttfamily,       % the size of the fonts that are used for the code
	% language=bash,
	% numbers=left,                   % where to put the line-numbers
	% numberstyle=\footnotesize,      % the size of the fonts that are used for the line-numbers
	% numberstyle=\tiny,
	% stepnumber=5,                  	% the step between two line-numbers. If it is 1 each line will be numbered
	% numbersep=5pt,                  % how far the line-numbers are from the code
	% showspaces=false,            	% show spaces adding particular underscores
	showstringspaces=false,      	% underline spaces within strings
	% showtabs=true,                 % show tabs within strings adding particular underscores
	frame=single,           		% adds a frame around the code
	tabsize=4,          			% sets default tabsize to 2 spaces
	breaklines=true,       			% sets automatic line breaking
	postbreak=\raisebox{0ex}[0ex][0ex]{\ensuremath{\color{red}\hookrightarrow\space}},
	showlines=true
}


% 命令重定义 
\renewcommand{\refname}{\bfseries{参~考~文~献}} 	%将Reference改为参考文献(用于 article)
% \renewcommand{\bibname}{参~考~文~献}	%将bibiography改为参考文献(用于 book)
\renewcommand{\baselinestretch}{1.38} 	%设置行间距
